\documentclass[]{tufte-handout}

% ams
\usepackage{amssymb,amsmath}

\usepackage{ifxetex,ifluatex}
\usepackage{fixltx2e} % provides \textsubscript
\ifnum 0\ifxetex 1\fi\ifluatex 1\fi=0 % if pdftex
  \usepackage[T1]{fontenc}
  \usepackage[utf8]{inputenc}
\else % if luatex or xelatex
  \makeatletter
  \@ifpackageloaded{fontspec}{}{\usepackage{fontspec}}
  \makeatother
  \defaultfontfeatures{Ligatures=TeX,Scale=MatchLowercase}
  \makeatletter
  \@ifpackageloaded{soul}{
     \renewcommand\allcapsspacing[1]{{\addfontfeature{LetterSpace=15}#1}}
     \renewcommand\smallcapsspacing[1]{{\addfontfeature{LetterSpace=10}#1}}
   }{}
  \makeatother

\fi

% graphix
\usepackage{graphicx}
\setkeys{Gin}{width=\linewidth,totalheight=\textheight,keepaspectratio}

% booktabs
\usepackage{booktabs}

% url
\usepackage{url}

% hyperref
\usepackage{hyperref}

% units.
\usepackage{units}


\setcounter{secnumdepth}{-1}

% citations


% pandoc syntax highlighting
\usepackage{color}
\usepackage{fancyvrb}
\newcommand{\VerbBar}{|}
\newcommand{\VERB}{\Verb[commandchars=\\\{\}]}
\DefineVerbatimEnvironment{Highlighting}{Verbatim}{commandchars=\\\{\}}
% Add ',fontsize=\small' for more characters per line
\newenvironment{Shaded}{}{}
\newcommand{\AlertTok}[1]{\textcolor[rgb]{1.00,0.00,0.00}{\textbf{#1}}}
\newcommand{\AnnotationTok}[1]{\textcolor[rgb]{0.38,0.63,0.69}{\textbf{\textit{#1}}}}
\newcommand{\AttributeTok}[1]{\textcolor[rgb]{0.49,0.56,0.16}{#1}}
\newcommand{\BaseNTok}[1]{\textcolor[rgb]{0.25,0.63,0.44}{#1}}
\newcommand{\BuiltInTok}[1]{#1}
\newcommand{\CharTok}[1]{\textcolor[rgb]{0.25,0.44,0.63}{#1}}
\newcommand{\CommentTok}[1]{\textcolor[rgb]{0.38,0.63,0.69}{\textit{#1}}}
\newcommand{\CommentVarTok}[1]{\textcolor[rgb]{0.38,0.63,0.69}{\textbf{\textit{#1}}}}
\newcommand{\ConstantTok}[1]{\textcolor[rgb]{0.53,0.00,0.00}{#1}}
\newcommand{\ControlFlowTok}[1]{\textcolor[rgb]{0.00,0.44,0.13}{\textbf{#1}}}
\newcommand{\DataTypeTok}[1]{\textcolor[rgb]{0.56,0.13,0.00}{#1}}
\newcommand{\DecValTok}[1]{\textcolor[rgb]{0.25,0.63,0.44}{#1}}
\newcommand{\DocumentationTok}[1]{\textcolor[rgb]{0.73,0.13,0.13}{\textit{#1}}}
\newcommand{\ErrorTok}[1]{\textcolor[rgb]{1.00,0.00,0.00}{\textbf{#1}}}
\newcommand{\ExtensionTok}[1]{#1}
\newcommand{\FloatTok}[1]{\textcolor[rgb]{0.25,0.63,0.44}{#1}}
\newcommand{\FunctionTok}[1]{\textcolor[rgb]{0.02,0.16,0.49}{#1}}
\newcommand{\ImportTok}[1]{#1}
\newcommand{\InformationTok}[1]{\textcolor[rgb]{0.38,0.63,0.69}{\textbf{\textit{#1}}}}
\newcommand{\KeywordTok}[1]{\textcolor[rgb]{0.00,0.44,0.13}{\textbf{#1}}}
\newcommand{\NormalTok}[1]{#1}
\newcommand{\OperatorTok}[1]{\textcolor[rgb]{0.40,0.40,0.40}{#1}}
\newcommand{\OtherTok}[1]{\textcolor[rgb]{0.00,0.44,0.13}{#1}}
\newcommand{\PreprocessorTok}[1]{\textcolor[rgb]{0.74,0.48,0.00}{#1}}
\newcommand{\RegionMarkerTok}[1]{#1}
\newcommand{\SpecialCharTok}[1]{\textcolor[rgb]{0.25,0.44,0.63}{#1}}
\newcommand{\SpecialStringTok}[1]{\textcolor[rgb]{0.73,0.40,0.53}{#1}}
\newcommand{\StringTok}[1]{\textcolor[rgb]{0.25,0.44,0.63}{#1}}
\newcommand{\VariableTok}[1]{\textcolor[rgb]{0.10,0.09,0.49}{#1}}
\newcommand{\VerbatimStringTok}[1]{\textcolor[rgb]{0.25,0.44,0.63}{#1}}
\newcommand{\WarningTok}[1]{\textcolor[rgb]{0.38,0.63,0.69}{\textbf{\textit{#1}}}}

% longtable

% multiplecol
\usepackage{multicol}

% strikeout
\usepackage[normalem]{ulem}

% morefloats
\usepackage{morefloats}


% tightlist macro required by pandoc >= 1.14
\providecommand{\tightlist}{%
  \setlength{\itemsep}{0pt}\setlength{\parskip}{0pt}}

% title / author / date
\title{Practical Machine Learning Course Project}
\author{Jessica Dyer}
\date{2/21/2021}

\usepackage{booktabs}
\usepackage{longtable}
\usepackage{array}
\usepackage{multirow}
\usepackage{wrapfig}
\usepackage{float}
\usepackage{colortbl}
\usepackage{pdflscape}
\usepackage{tabu}
\usepackage{threeparttable}
\usepackage{threeparttablex}
\usepackage[normalem]{ulem}
\usepackage{makecell}
\usepackage{xcolor}

\begin{document}

\maketitle




\begin{Shaded}
\begin{Highlighting}[]
\NormalTok{packages <-}\StringTok{ }\KeywordTok{c}\NormalTok{(}\StringTok{"dplyr"}\NormalTok{, }\StringTok{"ggplot2"}\NormalTok{, }\StringTok{"tidyverse"}\NormalTok{, }\StringTok{"caret"}\NormalTok{, }
              \StringTok{"Hmisc"}\NormalTok{, }\StringTok{"tibble"}\NormalTok{, }\StringTok{"kableExtra"}\NormalTok{, }\StringTok{"here"}\NormalTok{, }\StringTok{"gtsummary"}\NormalTok{, }
              \StringTok{"lubridate"}\NormalTok{, }\StringTok{"stringr"}\NormalTok{, }\StringTok{"readr"}\NormalTok{, }\StringTok{"utils"}\NormalTok{, }\StringTok{"naniar"}\NormalTok{)}

\NormalTok{new.packages <-}\StringTok{ }\NormalTok{packages[}\OperatorTok{!}\NormalTok{(packages }\OperatorTok\StringTok{ }\KeywordTok{installed.packages}\NormalTok{()[,}\StringTok{"Package"}\NormalTok{])]}

\ControlFlowTok{if}\NormalTok{(}\KeywordTok{length}\NormalTok{(new.packages)) }\KeywordTok{install.packages}\NormalTok{(new.packages)}

\CommentTok{# Load packages}
\KeywordTok{invisible}\NormalTok{(}\KeywordTok{lapply}\NormalTok{(packages, library, }\DataTypeTok{character.only =} \OtherTok{TRUE}\NormalTok{))}

\NormalTok{train_url <-}\StringTok{ "https://d396qusza40orc.cloudfront.net/predmachlearn/pml-training.csv"}
  
\NormalTok{test_url <-}\StringTok{ "https://d396qusza40orc.cloudfront.net/predmachlearn/pml-testing.csv"}

\NormalTok{train <-}\StringTok{ }\KeywordTok{read.csv}\NormalTok{(train_url)}
\CommentTok{# test <- read.csv(test_url)}
\end{Highlighting}
\end{Shaded}

\href{https://jessica-dyer.github.io/practicalmachinelearning/}{Link to
Github Pages} \#\# Executive summary

\hypertarget{introduction}{%
\subsection{Introduction}\label{introduction}}

Using devices such as Jawbone Up, Nike FuelBand, and Fitbit it is now
possible to collect a large amount of data about personal activity
relatively inexpensively. These type of devices are part of the
quantified self movement -- a group of enthusiasts who take measurements
about themselves regularly to improve their health, to find patterns in
their behavior, or because they are tech geeks. One thing that people
regularly do is quantify how much of a particular activity they do, but
they rarely quantify how well they do it. In this project, our goal is
to use data from accelerometers on the belt, forearm, arm, and dumbell
of 6 participants. They were asked to perform barbell lifts correctly
and incorrectly in 5 different ways. More information is available from
the website here: \url{http://groupware.les.inf.puc-rio.br/har} (see the
section on the Weight Lifting Exercise Dataset).

\hypertarget{methods}{%
\subsection{Methods}\label{methods}}

\begin{enumerate}
\def\labelenumi{\arabic{enumi}.}
\tightlist
\item
  load data
\item
  set seed and split data
\item
  Google which accelerometer data best predicts quality of movement?
\end{enumerate}

\begin{Shaded}
\begin{Highlighting}[]
\CommentTok{# SPLIT THE DATA INTO TRAINING AND }\AlertTok{TESTING}\CommentTok{ USING `createDataPartition`}

\KeywordTok{set.seed}\NormalTok{(}\DecValTok{43929}\NormalTok{)}
\NormalTok{in_train <-}\StringTok{ }\KeywordTok{createDataPartition}\NormalTok{(}\DataTypeTok{y =}\NormalTok{ train}\OperatorTok{$}\NormalTok{classe, }
                                \DataTypeTok{p =} \FloatTok{.70}\NormalTok{, }
                                \DataTypeTok{list =} \OtherTok{FALSE}\NormalTok{)}

\NormalTok{train <-}\StringTok{ }\NormalTok{train[in_train, ]}

\KeywordTok{dim}\NormalTok{(train)}
\end{Highlighting}
\end{Shaded}

\begin{verbatim}
## [1] 13737   160
\end{verbatim}

\begin{Shaded}
\begin{Highlighting}[]
\CommentTok{#test <- data[-in_train, ]}
\end{Highlighting}
\end{Shaded}

\hypertarget{removing-coviariates}{%
\subsubsection{Removing coviariates}\label{removing-coviariates}}

Some variables have no variability. This dataset has 160 variables, so
in order to quickly remove some of these variables, we looked at which
variables is a near zero variance predictor. We will remove these
variables from our training dataset.

\begin{Shaded}
\begin{Highlighting}[]
\NormalTok{nzv <-}\StringTok{ }\KeywordTok{nearZeroVar}\NormalTok{(train, }\DataTypeTok{saveMetrics =} \OtherTok{TRUE}\NormalTok{)}

\NormalTok{nzv_vars <-}\StringTok{ }
\StringTok{  }\NormalTok{nzv }\OperatorTok\StringTok{ }\KeywordTok{filter}\NormalTok{(nzv }\OperatorTok{==}\StringTok{ "TRUE"}\NormalTok{)}
  
\NormalTok{remove_vars <-}\StringTok{ }\KeywordTok{row.names}\NormalTok{(nzv_vars)}
\end{Highlighting}
\end{Shaded}

We removed 4 variables as possible predictors with this method. - note:
do we want them to be removed completely? Or just not included as a
potential predictor?

\begin{Shaded}
\begin{Highlighting}[]
\NormalTok{train <-}\StringTok{ }\NormalTok{train[, }\OperatorTok{!}\KeywordTok{names}\NormalTok{(train) }\OperatorTok\StringTok{ }\NormalTok{remove_vars]}
\end{Highlighting}
\end{Shaded}

Next I want to look at which variables have the most correlation with
the outcome \texttt{classe}

Examine missingness

\begin{Shaded}
\begin{Highlighting}[]
\KeywordTok{n_var_miss}\NormalTok{(train)}
\end{Highlighting}
\end{Shaded}

\begin{verbatim}
## [1] 45
\end{verbatim}

\begin{Shaded}
\begin{Highlighting}[]
\KeywordTok{vis_miss}\NormalTok{(train, }
         \DataTypeTok{warn_large_data =} \OtherTok{FALSE}\NormalTok{)}
\end{Highlighting}
\end{Shaded}

\includegraphics{index_files/figure-latex/unnamed-chunk-4-1}

\begin{Shaded}
\begin{Highlighting}[]
\KeywordTok{gg_miss_var}\NormalTok{(train)}
\end{Highlighting}
\end{Shaded}

\includegraphics{index_files/figure-latex/unnamed-chunk-4-2}

\begin{Shaded}
\begin{Highlighting}[]
\KeywordTok{library}\NormalTok{(rattle)}
\end{Highlighting}
\end{Shaded}

\begin{verbatim}
## Warning: package 'rattle' was built under R version 4.0.3
\end{verbatim}

\begin{verbatim}
## Loading required package: bitops
\end{verbatim}

\begin{verbatim}
## Rattle: A free graphical interface for data science with R.
## Version 5.4.0 Copyright (c) 2006-2020 Togaware Pty Ltd.
## Type 'rattle()' to shake, rattle, and roll your data.
\end{verbatim}

\begin{Shaded}
\begin{Highlighting}[]
\NormalTok{fancyRpartPlot}
\end{Highlighting}
\end{Shaded}

\begin{verbatim}
## function (model, main = "", sub, caption, palettes, type = 2, 
##     ...) 
## {
##     if (!inherits(model, "rpart")) 
##         stop("The model object must be an rpart object. ", "Instead we found: ", 
##             paste(class(model), collapse = ", "), ".")
##     roundint <- !is.null(model$model)
##     if (missing(sub) & missing(caption)) {
##         sub <- paste("Rattle", format(Sys.time(), "%Y-%b-%d %H:%M:%S"), 
##             Sys.info()["user"])
##     }
##     else {
##         if (missing(sub)) 
##             sub <- caption
##     }
##     num.classes <- length(attr(model, "ylevels"))
##     default.palettes <- c("Greens", "Blues", "Oranges", "Purples", 
##         "Reds", "Greys")
##     if (missing(palettes)) 
##         palettes <- default.palettes
##     missed <- setdiff(1:6, seq(length(palettes)))
##     palettes <- c(palettes, default.palettes[missed])
##     numpals <- 6
##     palsize <- 5
##     pals <- c(RColorBrewer::brewer.pal(9, palettes[1])[1:5], 
##         RColorBrewer::brewer.pal(9, palettes[2])[1:5], RColorBrewer::brewer.pal(9, 
##             palettes[3])[1:5], RColorBrewer::brewer.pal(9, palettes[4])[1:5], 
##         RColorBrewer::brewer.pal(9, palettes[5])[1:5], RColorBrewer::brewer.pal(9, 
##             palettes[6])[1:5])
##     if (model$method == "class") {
##         yval2per <- -(1:num.classes) - 1
##         per <- apply(model$frame$yval2[, yval2per], 1, function(x) x[1 + 
##             x[1]])
##     }
##     else {
##         per <- model$frame$yval/max(model$frame$yval)
##     }
##     per <- as.numeric(per)
##     if (model$method == "class") 
##         col.index <- ((palsize * (model$frame$yval - 1) + trunc(pmin(1 + 
##             (per * palsize), palsize)))%%(numpals * palsize))
##     else col.index <- round(per * (palsize - 1)) + 1
##     col.index <- abs(col.index)
##     if (model$method == "class") 
##         extra <- 104
##     else extra <- 101
##     rpart.plot::prp(model, type = type, extra = extra, box.col = pals[col.index], 
##         nn = TRUE, varlen = 0, faclen = 0, shadow.col = "grey", 
##         fallen.leaves = TRUE, branch.lty = 3, roundint = roundint, 
##         ...)
##     title(main = main, sub = sub)
## }
## <bytecode: 0x000000003c25a8a8>
## <environment: namespace:rattle>
\end{verbatim}

\begin{Shaded}
\begin{Highlighting}[]
\CommentTok{# mod_fit <- train(classe ~ ., method = "rpart", data = train)}
\end{Highlighting}
\end{Shaded}

\hypertarget{results}{%
\subsection{Results}\label{results}}

\hypertarget{conclusion}{%
\subsection{Conclusion}\label{conclusion}}



\end{document}
